\chapter{Specyfikacja wymagań}
\section{User stories}
Aktorzy: gracz, właściciel pokoju, rysujący, zgadujący.
\subsection{Gracz}
Jako gracz...
\begin{center}
    \begin{tabular}{ | l || p{5.5cm} | p{5.5cm} |}
    \hline
     & \textbf{chcę...} & \textbf{aby...} \\ \hline \hline
    1 & mieć możliwość zarejestrowania konta & mieć wyłączny dostęp do swoich osiągnięć w grze \\ \hline
    2 & mieć możliwość zalogowania się na swoje konto & grać pod swoim nickiem i mieć dostęp do swoich osiągnięć \\ \hline
    3 & by moje konto było zapamiętane w grze & nie musieć się logować przy każdym uruchomieniu gry \\ \hline
    4 & mieć możliwość założenia pokoju do gry & móc samemu wybrać jego ustawienia \\ \hline
    5 & mieć możliwość dołączenia do wybranego przeze mnie pokoju & rozpocząć grę w pokoju, w którym są moi znajomi \\ \hline
    6 & mieć możliwość dołączenia do losowego pokoju & móc nie podejmować decyzji o wyborze pokoju \\ \hline
    7 & mieć możliwość zmienienia ustawień dźwięku & dopasować je do moich preferencji \\ \hline
    8 & mieć możliwość obejrzenia rankingu & porównać swoje osiągnięcia z innymi graczami \\ \hline
    9 & mieć możliwość sprawdzenia, kto jest twórcami gry & móc znaleźć inne projekty tych osób \\ \hline
    10 & mieć możliwość wyjścia z gry & zakończyć grę \\ \hline
    11 & słyszeć dźwięki podczas gry & być informowanym o zdarzeniach w Grze \\ \hline
    \end{tabular}
\end{center}

\subsection{Właściciel pokoju}
Jako właściciel pokoju...
\begin{center}
    \begin{tabular}{ | l || p{5.5cm} | p{5.5cm} |}
    \hline
     & \textbf{chcę...} & \textbf{aby...} \\ \hline \hline
    1 & mieć możliwość ustalenia limitu graczy w pokoju & uniknąć sytuacji w której zbyt długo czeka się na swój ruch \\ \hline
    2 & mieć możliwość założenia hasła do pokoju & mieli do niego dostęp jedynie gracze, którym podam hasło \\ \hline
    3 & mieć możliwość rozpoczęcia rozgrywki w Pokoju & zacząć grę \\ \hline
    4 & mieć możliwość nadania Pokojowi nazwy & był rozpoznawalny pośród innych \\ \hline
    \end{tabular}
\end{center}

\subsection{Rysujący}
Jako rysujący...
\begin{center}
    \begin{tabular}{ | l || p{5.5cm} | p{5.5cm} |}
    \hline
     & \textbf{chcę...} & \textbf{aby...} \\ \hline \hline
    1 & poznać hasło, które mam rysować na początku rysowania & wiedzieć, co rysować \\ \hline
    2 & mieć możliwość tworzenia brył na scenie & mieć kształt początkowy \\ \hline
    3 & mieć możliwość wybierania kształtu brył, które tworzę & tworzyć bardziej zróżnicowane konstrukcje \\ \hline
    4 & mieć możliwość wybierania koloru brył, które tworzę & tworzyć bardziej zróżnicowane konstrukcje \\ \hline
    5 & mieć możliwość przesuwania brył po scenie & tworzyć bardziej zróżnicowane konstrukcje \\ \hline
    6 & mieć możliwość wyczyszczenia sceny & móc zacząć rysować od początku \\ \hline
    7 & mieć możliwość obracania Sceny & móc obejrzeć rysunek z każdej strony \\ \hline
    8 & mieć możliwość tworzenia wklęśnięć w bryłach & tworzyć bardziej zróżnicowane konstrukcje \\ \hline
    9 & mieć możliwość tworzenia wzniesień na bryłach & tworzyć bardziej zróżnicowane konstrukcje \\ \hline
    10 & mieć możliwość dostosowywania promienia i ostrości tworzonych wklęśnięć i wzniesień & tworzyć bardziej zróżnicowane konstrukcje \\ \hline
    11 & widzieć czat & wiedzieć, czy mój rysunek naprowadza zgadujących na dobry trop \\ \hline
    12 & zakończyć rysowania tak szybko jak ktoś zgadnie hasło & runda mogła się skończyć \\ \hline
    13 & mieć ograniczony czas przeznaczony na rysowanie & nie rysować w nieskończoność \\ \hline
    \end{tabular}
\end{center}

\subsection{Zgadujący}
Jako zgadujący...
\begin{center}
    \begin{tabular}{ | l || p{5.5cm} | p{5.5cm} |}
    \hline
     & \textbf{chcę...} & \textbf{aby...} \\ \hline \hline
    1 & mieć możliwość obracania sceny niezależnie od rysującego & móc się przyjrzeć rysowanym bryłom \\ \hline
    2 & mieć możliwość pisania swoich odpowiedzi na chacie & serwer mógł określić, czy pasują one do zgadywanego hasła \\ \hline
    3 & widzieć chat i wszystkie wiadomości przesyłane w pokoju & poznać odpowiedzi innych graczy \\ \hline
    4 & widzieć punkty swoje i innych graczy znajdujących się w pokoju & był wiedzieć jakie są moje szanse na wygraną \\ \hline
    \end{tabular}
\end{center}